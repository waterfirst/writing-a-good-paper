% Options for packages loaded elsewhere
\PassOptionsToPackage{unicode}{hyperref}
\PassOptionsToPackage{hyphens}{url}
%

\documentclass[
  12pt]{iopart}

\usepackage{amsmath,amssymb}
\usepackage{iftex}
\ifPDFTeX
  \usepackage[T1]{fontenc}
  \usepackage[utf8]{inputenc}
  \usepackage{textcomp} % provide euro and other symbols
\else % if luatex or xetex
  \usepackage{unicode-math}
  \defaultfontfeatures{Scale=MatchLowercase}
  \defaultfontfeatures[\rmfamily]{Ligatures=TeX,Scale=1}
\fi
\usepackage{lmodern}
\ifPDFTeX\else  
    % xetex/luatex font selection
\fi
% Use upquote if available, for straight quotes in verbatim environments
\IfFileExists{upquote.sty}{\usepackage{upquote}}{}
\IfFileExists{microtype.sty}{% use microtype if available
  \usepackage[]{microtype}
  \UseMicrotypeSet[protrusion]{basicmath} % disable protrusion for tt fonts
}{}
\makeatletter
\@ifundefined{KOMAClassName}{% if non-KOMA class
  \IfFileExists{parskip.sty}{%
    \usepackage{parskip}
  }{% else
    \setlength{\parindent}{0pt}
    \setlength{\parskip}{6pt plus 2pt minus 1pt}}
}{% if KOMA class
  \KOMAoptions{parskip=half}}
\makeatother
\usepackage{xcolor}
\setlength{\emergencystretch}{3em} % prevent overfull lines
\setcounter{secnumdepth}{5}
% Make \paragraph and \subparagraph free-standing
\ifx\paragraph\undefined\else
  \let\oldparagraph\paragraph
  \renewcommand{\paragraph}[1]{\oldparagraph{#1}\mbox{}}
\fi
\ifx\subparagraph\undefined\else
  \let\oldsubparagraph\subparagraph
  \renewcommand{\subparagraph}[1]{\oldsubparagraph{#1}\mbox{}}
\fi

\usepackage{color}
\usepackage{fancyvrb}
\newcommand{\VerbBar}{|}
\newcommand{\VERB}{\Verb[commandchars=\\\{\}]}
\DefineVerbatimEnvironment{Highlighting}{Verbatim}{commandchars=\\\{\}}
% Add ',fontsize=\small' for more characters per line
\usepackage{framed}
\definecolor{shadecolor}{RGB}{241,243,245}
\newenvironment{Shaded}{\begin{snugshade}}{\end{snugshade}}
\newcommand{\AlertTok}[1]{\textcolor[rgb]{0.68,0.00,0.00}{#1}}
\newcommand{\AnnotationTok}[1]{\textcolor[rgb]{0.37,0.37,0.37}{#1}}
\newcommand{\AttributeTok}[1]{\textcolor[rgb]{0.40,0.45,0.13}{#1}}
\newcommand{\BaseNTok}[1]{\textcolor[rgb]{0.68,0.00,0.00}{#1}}
\newcommand{\BuiltInTok}[1]{\textcolor[rgb]{0.00,0.23,0.31}{#1}}
\newcommand{\CharTok}[1]{\textcolor[rgb]{0.13,0.47,0.30}{#1}}
\newcommand{\CommentTok}[1]{\textcolor[rgb]{0.37,0.37,0.37}{#1}}
\newcommand{\CommentVarTok}[1]{\textcolor[rgb]{0.37,0.37,0.37}{\textit{#1}}}
\newcommand{\ConstantTok}[1]{\textcolor[rgb]{0.56,0.35,0.01}{#1}}
\newcommand{\ControlFlowTok}[1]{\textcolor[rgb]{0.00,0.23,0.31}{#1}}
\newcommand{\DataTypeTok}[1]{\textcolor[rgb]{0.68,0.00,0.00}{#1}}
\newcommand{\DecValTok}[1]{\textcolor[rgb]{0.68,0.00,0.00}{#1}}
\newcommand{\DocumentationTok}[1]{\textcolor[rgb]{0.37,0.37,0.37}{\textit{#1}}}
\newcommand{\ErrorTok}[1]{\textcolor[rgb]{0.68,0.00,0.00}{#1}}
\newcommand{\ExtensionTok}[1]{\textcolor[rgb]{0.00,0.23,0.31}{#1}}
\newcommand{\FloatTok}[1]{\textcolor[rgb]{0.68,0.00,0.00}{#1}}
\newcommand{\FunctionTok}[1]{\textcolor[rgb]{0.28,0.35,0.67}{#1}}
\newcommand{\ImportTok}[1]{\textcolor[rgb]{0.00,0.46,0.62}{#1}}
\newcommand{\InformationTok}[1]{\textcolor[rgb]{0.37,0.37,0.37}{#1}}
\newcommand{\KeywordTok}[1]{\textcolor[rgb]{0.00,0.23,0.31}{#1}}
\newcommand{\NormalTok}[1]{\textcolor[rgb]{0.00,0.23,0.31}{#1}}
\newcommand{\OperatorTok}[1]{\textcolor[rgb]{0.37,0.37,0.37}{#1}}
\newcommand{\OtherTok}[1]{\textcolor[rgb]{0.00,0.23,0.31}{#1}}
\newcommand{\PreprocessorTok}[1]{\textcolor[rgb]{0.68,0.00,0.00}{#1}}
\newcommand{\RegionMarkerTok}[1]{\textcolor[rgb]{0.00,0.23,0.31}{#1}}
\newcommand{\SpecialCharTok}[1]{\textcolor[rgb]{0.37,0.37,0.37}{#1}}
\newcommand{\SpecialStringTok}[1]{\textcolor[rgb]{0.13,0.47,0.30}{#1}}
\newcommand{\StringTok}[1]{\textcolor[rgb]{0.13,0.47,0.30}{#1}}
\newcommand{\VariableTok}[1]{\textcolor[rgb]{0.07,0.07,0.07}{#1}}
\newcommand{\VerbatimStringTok}[1]{\textcolor[rgb]{0.13,0.47,0.30}{#1}}
\newcommand{\WarningTok}[1]{\textcolor[rgb]{0.37,0.37,0.37}{\textit{#1}}}

\providecommand{\tightlist}{%
  \setlength{\itemsep}{0pt}\setlength{\parskip}{0pt}}\usepackage{longtable,booktabs,array}
\usepackage{calc} % for calculating minipage widths
% Correct order of tables after \paragraph or \subparagraph
\usepackage{etoolbox}
\makeatletter
\patchcmd\longtable{\par}{\if@noskipsec\mbox{}\fi\par}{}{}
\makeatother
% Allow footnotes in longtable head/foot
\IfFileExists{footnotehyper.sty}{\usepackage{footnotehyper}}{\usepackage{footnote}}
\makesavenoteenv{longtable}
\usepackage{graphicx}
\makeatletter
\def\maxwidth{\ifdim\Gin@nat@width>\linewidth\linewidth\else\Gin@nat@width\fi}
\def\maxheight{\ifdim\Gin@nat@height>\textheight\textheight\else\Gin@nat@height\fi}
\makeatother
% Scale images if necessary, so that they will not overflow the page
% margins by default, and it is still possible to overwrite the defaults
% using explicit options in \includegraphics[width, height, ...]{}
\setkeys{Gin}{width=\maxwidth,height=\maxheight,keepaspectratio}
% Set default figure placement to htbp
\makeatletter
\def\fps@figure{htbp}
\makeatother

\usepackage{orcidlink}
\definecolor{mypink}{RGB}{219, 48, 122}
\usepackage{fvextra}
\DefineVerbatimEnvironment{Highlighting}{Verbatim}{breaklines,commandchars=\\\{\}}
\makeatletter
\@ifpackageloaded{caption}{}{\usepackage{caption}}
\AtBeginDocument{%
\ifdefined\contentsname
  \renewcommand*\contentsname{Table of contents}
\else
  \newcommand\contentsname{Table of contents}
\fi
\ifdefined\listfigurename
  \renewcommand*\listfigurename{List of Figures}
\else
  \newcommand\listfigurename{List of Figures}
\fi
\ifdefined\listtablename
  \renewcommand*\listtablename{List of Tables}
\else
  \newcommand\listtablename{List of Tables}
\fi
\ifdefined\figurename
  \renewcommand*\figurename{Figure}
\else
  \newcommand\figurename{Figure}
\fi
\ifdefined\tablename
  \renewcommand*\tablename{Table}
\else
  \newcommand\tablename{Table}
\fi
}
\@ifpackageloaded{float}{}{\usepackage{float}}
\floatstyle{ruled}
\@ifundefined{c@chapter}{\newfloat{codelisting}{h}{lop}}{\newfloat{codelisting}{h}{lop}[chapter]}
\floatname{codelisting}{Listing}
\newcommand*\listoflistings{\listof{codelisting}{List of Listings}}
\makeatother
\makeatletter
\makeatother
\makeatletter
\@ifpackageloaded{caption}{}{\usepackage{caption}}
\@ifpackageloaded{subcaption}{}{\usepackage{subcaption}}
\makeatother
\ifLuaTeX
  \usepackage{selnolig}  % disable illegal ligatures
\fi
\nocite{*}
\IfFileExists{bookmark.sty}{\usepackage{bookmark}}{\usepackage{hyperref}}
\IfFileExists{xurl.sty}{\usepackage{xurl}}{} % add URL line breaks if available
\urlstyle{same} % disable monospaced font for URLs
\hypersetup{
  pdftitle={So you want to publish in IOP? Here's a template for you.},
  pdfauthor={John Smith; Jane Doe; Michael Johnson},
  pdfkeywords={extinction, deforestation, species, breakdown, survival},
  hidelinks,
  pdfcreator={LaTeX via pandoc}}


\usepackage[numbers,square,sort&compress]{natbib}
\renewcommand{\bibsection}{\section*{References}}
\bibliographystyle{iopart-num}

\begin{document}

\title[IOP Template Format]{So you want to publish in IOP? Here's a
template for you.}

\author{John Smith$^1$\orcidlink{0000-0000-0000-0000}, Jane
Doe$^1$$^,$$^2$\orcidlink{0000-0000-0000-0000} and Michael
Johnson$^3$\orcidlink{0000-0000-0000-0000}}

    \address{$^1$ Department of Biodiversity Conservation, Ecological
Research Institute, 123 Willow Road, Anytown, UK NW 123}
    \address{$^2$ Department of Ecosystem Management, University of
Environmental Science, 456 Birch Lane, Anycity, UK NR 543}
    \address{$^3$ Department of Renewable Energy and Carbon
Sequestration, Institute of Climate Change and Sustainability, 789 Oak
Street, Anybury, UK BT 987}

\ead{michaeljohnson@fakeemail.com}

\begin{abstract}
The ecological emergency refers to the current state of the global
environment, characterized by unprecedented declines in biodiversity,
rapid loss of natural habitats, and increasing frequency of extreme
weather events. The root cause of the ecological emergency is human
activity, specifically the overconsumption of resources and the release
of greenhouse gases. The situation is dire, with scientists warning that
up to one million species are at risk of extinction due to human
activities. The ecological emergency also poses a threat to human
well-being as it disrupts the provision of vital ecosystem services such
as air and water purification, pollination, and climate regulation. The
time to act is now, as the window of opportunity to prevent the worst
outcomes of the ecological emergency is rapidly closing. It is crucial
that immediate and ambitious actions are taken at the global, national,
and local levels to reduce greenhouse gas emissions, protect
biodiversity, and restore natural habitats. This requires a
transformation of our economic and societal systems towards
sustainability.
\end{abstract}

\noindent{\it Keywords\/}: extinction, deforestation, species, breakdown, survival

\submitto{\ERL}

\maketitle


\section{Introduction}\label{introduction}

\subsection{ChatGPT, what are the basics of
markdown?}\label{chatgpt-what-are-the-basics-of-markdown}

Markdown is a lightweight markup language that is used to format plain
text documents. It is designed to be simple and easy to read, and it
uses a set of characters and symbols to indicate formatting such as
headings, lists, and emphasis. Some of the basics of Markdown include:

\begin{itemize}
\item
  Headings are indicated by the pound sign (\#) followed by a space. The
  number of pound signs used indicates the level of the heading, with
  one pound sign indicating the highest level heading and six pound
  signs indicating the lowest level heading.
\item
  Lists are created by using the dash (-) or asterisk (*) followed by a
  space for each list item.
\item
  Emphasis can be indicated by using asterisks (*) or underscores (\_)
  around a word or phrase. One asterisk or underscore is used for italic
  emphasis, and two asterisks or underscores are used for bold emphasis.
\item
  Links can be added by using the brackets {[}{]} followed by the link
  text, then the brackets () followed by the link URL.
\item
  Images can be added by using the exclamation mark ! followed by the
  brackets {[}{]} for the image's alternate text, then the brackets ()
  followed by the image URL.
\item
  Code blocks can be indicated by using three back-ticks (```) before
  and after the code snippet.
\end{itemize}

Markdown is a versatile and flexible language, and there are many more
elements you can add, but these are the basics to get you started.
Additionally, there are different flavors of markdown, so some elements
may vary depending on the implementation.

For more ``official'' guidance:
\url{https://quarto.org/docs/authoring/markdown-basics.html}.

\subsection{And what about the basics of
Quarto?}\label{sec-quarto-basics}

Quarto is a strategy board game for two players. It was invented by
Swiss mathematician Blaise Müller\ldots{}

..I'm going to stop you there.

Basics of Quarto: \url{https://quarto.org/docs/guide/}.

\section{Using this template}\label{using-this-template}

\subsection{\texorpdfstring{Format options and differences between
\texttt{iop-html} and
\texttt{iop-pdf}}{Format options and differences between iop-html and iop-pdf}}\label{sec-formopt}

\begin{itemize}
\tightlist
\item
  For the HTML format, we are using Pandoc citeproc to include the
  bibliography. Here \texttt{reference-section-title} controls the title
  for the chapter that will be used (although, specific to this
  template, we also use citeproc for the PDF to get Harvard style
  citations).
\item
  For the PDF format, \texttt{natbib} is used by default and the
  bibliography is included with a title by the {\LaTeX} template.
\end{itemize}

This format provides a number of custom YAML header options to control
the PDF format:

\begin{Shaded}
\begin{Highlighting}[]
\CommentTok{\# replaces titlepage class option}
\FunctionTok{titlepage}\KeywordTok{:}\AttributeTok{ }\CharTok{true}
\CommentTok{\# set to true to use the orcidlink package}
\CommentTok{\# to display linked orcid logos}
\CommentTok{\# not required by IOP but fancy...}
\FunctionTok{display{-}orcids}\KeywordTok{:}\AttributeTok{ }\CharTok{true}
\CommentTok{\# use journal macros set out in the IOP style guide}
\CommentTok{\# to add which journal an article is targeting}
\CommentTok{\# e.g. \textbackslash{}ERL = Environmental Research Letters}
\FunctionTok{submitted{-}to}\KeywordTok{:}\AttributeTok{ \textbackslash{}ERL}
\CommentTok{\# test your article in two column form}
\CommentTok{\# NB: does not work well with code chunk outputs}
\FunctionTok{twocol}\KeywordTok{:}\AttributeTok{ }\CharTok{true}
\CommentTok{\# sets Vancouver numeric citation style}
\CommentTok{\# comment out both iop{-}vancouver and cite{-}method}
\CommentTok{\# for Harvard author year citation style}
\FunctionTok{iop{-}vancouver}\KeywordTok{:}\AttributeTok{ }\CharTok{true}
\FunctionTok{iop{-}pdf}\KeywordTok{:}
\AttributeTok{  }\FunctionTok{cite{-}method}\KeywordTok{:}\AttributeTok{ natbib}
\CommentTok{\# see iopart.cls for option source code}
\CommentTok{\# possible options (separate with comma):}
\CommentTok{\# a4paper, letterpaper, 10pt, 12pt, draft, final}
\FunctionTok{classoptions}\KeywordTok{:}\AttributeTok{ }\KeywordTok{[]}
\end{Highlighting}
\end{Shaded}

For the most part, it \emph{should not be necessary} to modify the
\texttt{\_extension.yml} file, except maybe to switch the referencing
style for the HTML output.

\subsection{IOP journal and format
options}\label{iop-journal-and-format-options}

\begin{longtable}[]{@{}
  >{\raggedright\arraybackslash}p{(\columnwidth - 6\tabcolsep) * \real{0.4000}}
  >{\raggedright\arraybackslash}p{(\columnwidth - 6\tabcolsep) * \real{0.1000}}
  >{\raggedright\arraybackslash}p{(\columnwidth - 6\tabcolsep) * \real{0.4000}}
  >{\raggedright\arraybackslash}p{(\columnwidth - 6\tabcolsep) * \real{0.1000}}@{}}
\caption{Journals to which this document applies, and macros for the
abbreviated journal names.}\label{tbl-macros}\tabularnewline
\toprule\noalign{}
\begin{minipage}[b]{\linewidth}\raggedright
Short form of journal title
\end{minipage} & \begin{minipage}[b]{\linewidth}\raggedright
Macro name
\end{minipage} & \begin{minipage}[b]{\linewidth}\raggedright
Short form of journal title
\end{minipage} & \begin{minipage}[b]{\linewidth}\raggedright
Macro name
\end{minipage} \\
\midrule\noalign{}
\endfirsthead
\toprule\noalign{}
\begin{minipage}[b]{\linewidth}\raggedright
Short form of journal title
\end{minipage} & \begin{minipage}[b]{\linewidth}\raggedright
Macro name
\end{minipage} & \begin{minipage}[b]{\linewidth}\raggedright
Short form of journal title
\end{minipage} & \begin{minipage}[b]{\linewidth}\raggedright
Macro name
\end{minipage} \\
\midrule\noalign{}
\endhead
\bottomrule\noalign{}
\endlastfoot
2D Mater. & \textbackslash TDM & Mater. Res. Express &
\textbackslash MRE \\
Biofabrication & \textbackslash BF & Meas. Sci.
Technol.\textsuperscript{c} & \textbackslash MST \\
Bioinspir. Biomim. & \textbackslash BB & Methods Appl. Fluoresc. &
\textbackslash MAF \\
Biomed. Mater. & \textbackslash BMM & Modelling Simul. Mater. Sci. Eng.
& \textbackslash MSMSE \\
Class. Quantum Grav. & \textbackslash CQG & Nucl. Fusion &
\textbackslash NF \\
Comput. Sci. Disc. & \textbackslash CSD & New J. Phys. &
\textbackslash NJP \\
Environ. Res. Lett. & \textbackslash ERL &
Nonlinearity\textsuperscript{a,b} & \textbackslash NL \\
Eur. J. Phys. & \textbackslash EJP & Nanotechnology &
\textbackslash NT \\
Inverse Problems & \textbackslash IP & Phys. Biol.\textsuperscript{c} &
\textbackslash PB \\
J. Breath Res. & \textbackslash JBR & Phys. Educ.\textsuperscript{a} &
\textbackslash PED \\
J. Geophys. Eng.\textsuperscript{d} & \textbackslash JGE & Physiol.
Meas.\textsuperscript{c,d,e} & \textbackslash PM \\
J. Micromech. Microeng. & \textbackslash JMM & Phys. Med.
Biol.\textsuperscript{c,d,e} & \textbackslash PMB \\
J. Neural Eng.\textsuperscript{c} & \textbackslash JNE & Plasma Phys.
Control. Fusion & \textbackslash PPCF \\
J. Opt. & \textbackslash JOPT & Phys. Scr. & \textbackslash PS \\
J. Phys. A: Math. Theor. & \textbackslash jpa & Plasma Sources Sci.
Technol. & \textbackslash PSST \\
J. Phys. B: At. Mol. Opt. Phys. & \textbackslash jpb & Rep.~Prog.
Phys.\textsuperscript{e} & \textbackslash RPP \\
J. Phys: Condens. Matter & \textbackslash JPCM & Semicond. Sci. Technol.
& \textbackslash SST \\
J. Phys. D: Appl. Phys. & \textbackslash JPD & Smart Mater. Struct. &
\textbackslash SMS \\
J. Phys. G: Nucl. Part. Phys. & \textbackslash jpg & Supercond. Sci.
Technol. & \textbackslash SUST \\
J. Radiol. Prot.\textsuperscript{a} & \textbackslash JRP & Surf.
Topogr.: Metrol. Prop. & \textbackslash STMP \\
Metrologia & \textbackslash MET & Transl. Mater. Res. &
\textbackslash TMR \\
\end{longtable}

\textsuperscript{a}UK spelling is required; \textsuperscript{b}MSC
classification numbers are required; \textsuperscript{c}titles of
articles are required in journal references;
\textsuperscript{d}Harvard-style references must be used (see Section
Section~\ref{sec-formopt}); \textsuperscript{e}final page numbers of
articles are required in journal references.

This template has been written for the submission of an \emph{article}
to one of the journals laid out in Table~\ref{tbl-macros}. For other
types of content for submission, the \texttt{\textbackslash{}title}
{\LaTeX} control sequence on line 2 of \texttt{partials/before-body.tex}
should be replaced with:

\begin{itemize}
\tightlist
\item
  \texttt{\textbackslash{}paper} for a paper;
\item
  \texttt{\textbackslash{}letter} for a \emph{Letter to the Editor};
\item
  \texttt{\textbackslash{}ftc} for a \emph{Fast Track Communication};
\item
  \texttt{\textbackslash{}rapid} for a \emph{Rapid Communication};
\item
  \texttt{\textbackslash{}comment} for a \emph{Comment};
\item
  \texttt{\textbackslash{}topical} for a \emph{Topical Review};
\item
  \texttt{\textbackslash{}review} for a review article;
\item
  \texttt{\textbackslash{}note} for a \emph{Note}; and
\item
  \texttt{\textbackslash{}prelim} for a \emph{Preliminary Communication}
\end{itemize}

\subsection{Shortcode demo}\label{sec-shortcode}

PDF are rendered using {\LaTeX} but it is best if one can use a Markdown
syntax for cross format support.

{} used in source is a shortcode syntax to print ``LaTeX'' in fancy text
where the shortcode is included in the extension folder
\texttt{\_extensions}.

Alternatively, Quarto supports a
\href{https://quarto.org/docs/extensions/shortcodes.html}{number of
shortcodes} natively. For example, {}

\newpage{}

\ldots inserts a page break.

\subsection{Code chunk}\label{sec-chunks}

This format hide chunks by default as option has been set in
\texttt{\_extension.yml} file.

But you can set \texttt{echo} option to \texttt{true} locally in the
chunk

\begin{Shaded}
\begin{Highlighting}[]
\FunctionTok{head}\NormalTok{(iris)}
\end{Highlighting}
\end{Shaded}

\begin{verbatim}
  Sepal.Length Sepal.Width Petal.Length Petal.Width Species
1          5.1         3.5          1.4         0.2  setosa
2          4.9         3.0          1.4         0.2  setosa
3          4.7         3.2          1.3         0.2  setosa
4          4.6         3.1          1.5         0.2  setosa
5          5.0         3.6          1.4         0.2  setosa
6          5.4         3.9          1.7         0.4  setosa
\end{verbatim}

\subsection{Text color}\label{sec-summary}

Our format makes applying color on inline text possible using the
\texttt{{[}content{]}\{color=\textless{}name\textgreater{}\}} syntax.
Let's see an example.

Here we are using a special feature of our format which is the coloring
because \textcolor{mypink}{pink is a \textbf{nice} color}.

This is possible thanks to the Lua Filter included in the custom
extension format.

\subsection{Using references}\label{using-references}

I did not read this book \citep{Brown2019} -- coded
\texttt{{[}@Brown2019{]}} -- but it must be interesting. We can add all
references in the our bibliography file, even if not cited in text, with
the following YAML code:

\begin{Shaded}
\begin{Highlighting}[]
\FunctionTok{nocite}\KeywordTok{: }\CharTok{|}
\NormalTok{  \textquotesingle{}@*\textquotesingle{}}
\end{Highlighting}
\end{Shaded}

\subsection{Quick maths}\label{quick-maths}

Two plus two is four, minus one that's three -- quick maths.

{\LaTeX} and Quarto handle maths very well. This can be either inline,
such as \(E = mc^{2}\), or display maths like Equation~\ref{eq-demo} and
Equation~\ref{eq-demo2} below:

\begin{equation}\phantomsection\label{eq-demo}{
P(e) - {n \choose k} p^k (2-p)^{n-k}
}\end{equation}

\begin{equation}\phantomsection\label{eq-demo2}{
\frac{\partial \mathrm C}{ \partial \mathrm t } + \frac{1}{2}\sigma^{2} \mathrm S^{2}
\frac{\partial^{2} \mathrm C}{\partial \mathrm C^2}
  + \mathrm r \mathrm S \frac{\partial \mathrm C}{\partial \mathrm S}\ =
  \mathrm r \mathrm C 
}\end{equation}

Please see the IOP {\LaTeX} style guide for more about mathematical
equations, noting specifically p2:

\begin{quote}
``Also note that there is an incompatibility between amsmath.sty and
iopart.cls which cannot be completely worked around. If your article
relies on commands in amsmath.sty that are not available in iopart.cls,
you may wish to consider using a different class file.''
\end{quote}

In order to avoid package clashes in this template, lines 788-789 of
\texttt{iopart.cls} were commented out.

\subsection{Figures}\label{sec-figures}

Figures can be included as normal using markdown syntax:

\begin{figure}[H]

{\centering \includegraphics{fig.png}

}

\caption{A comic book cover of the publishing company IOP Publishing
reimagined as a terrifying monster.}

\end{figure}%

See the link in Section~\ref{sec-quarto-basics} for more on figure
embedding. For IOP, figures can be included in the body of the text or
grouped together at the end of the document, although the former
requires less fiddling using this template.

\section*{Acknowledgements}\label{sec-ack}
\addcontentsline{toc}{section}{Acknowledgements}

IOP acknowledgements go here: unnumbered, after the last numbered
section and before the appendices or references. Acknowledgement of
funding should also be included here.

\section{Adding appendices}\label{adding-appendices}

The shortcodes {} and {} have been added for convenience. Appendices
must be labelled A, B, C etc. and placed here: after the
acknowledgements, before references.

\appendix

\section{PDF format}\label{pdf-format}

For only one appendix, {} inserts the section heading ``Appendix'', and
resets numbering so any equations and figures are correctly numbered.
For more than one appendix (as here), {} can be used: it will open the
{\LaTeX} \texttt{\textbackslash{}appendix} environment, allowing use of
normal markdown section headers for each appendix: \texttt{\#\ Data},
\texttt{\#\ Tables}, etc.

\begin{equation}\phantomsection\label{eq-app}{
\sqrt{x^2+1}
}\end{equation}

\section{HTML format}\label{html-format}

HTML does not handle appendix numbering as well as {\LaTeX}. However,
appendices can be included in the HTML appendix by adding
\texttt{\{.appendix\}} after the section header,
e.g.~\texttt{\#\ Data\ \{.appendix\}}.


  \bibliography{bibliography.bib}


\end{document}
